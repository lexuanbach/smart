\begin{figure}
	\centering
	\begin{tikzpicture}[% 	
	every text node part/.style={align=center},
	->,
	shorten >=2pt,
	>=stealth,
	node distance=6em,
	com/.style={%
		rectangle,
		rounded corners,
		fill=gray!40,
		minimum width=3em,
		minimum height=1em,
		align=center,
		draw
	},
	sub/.style={%
		rectangle,
		rounded corners,
		minimum width=3em,
		minimum height=1em,
		draw
	},
	data/.style={%
		rectangle,
		minimum width=3em,
		minimum height=1em,
	},
	res/.style={%
		rectangle,
		dashed,
		minimum width=3em,
		minimum height=1em,
		draw
	}
	] 

	\node[com,text width=2cm, text height = 0.3cm] (transformer){\texttt{transformer}}; 
	\node[com,text width=2cm, text height = 0.3cm] (matcher)[right=of transformer]{\texttt{matcher}}; 
	\node[com,text width=2cm, text height = 0.3cm] (verifier)[right=of matcher, xshift=3em]{\texttt{verifier}};   
	\node[data] (input) [above=of transformer,yshift=-2em] {function f};
	\node[data] (trace)[above=of matcher,yshift=-2em]{vunerable trace $T_v$}; 
	\node[res] (result)[above=of verifier,yshift=-2em]{report bug};      
	\draw  (input) -- (transformer);    
	\draw  (trace) -- (matcher);   
	\draw (transformer) --node[data,above] {trace $T_f$} (matcher);  
	  \path[every node/.style={}]
	  %(verifier) edge[bend left=30] node[below]  {V is infeasible} (matcher)
    %(matcher) edge[bend left=30] node[above]  {matched with new \\ attack vector %V} (verifier)
    (matcher) edge[above] node {attack trace \\ vectors $\{V_i\}_{i=1}^n$} (verifier)
    (verifier) edge[right] node {some $V_i$ \\ is feasible} (result);
	\end{tikzpicture}
	\caption{Our framework for checking vulnerabilities in Ethereum smart contracts.}\label{fig:framework}	
\end{figure}

We illustrate the high-level of our framework in Fig.~\ref{fig:framework}. It has three main components: \texttt{transformer}, \texttt{matcher} and \texttt{verifier}. To check whether a smart contract function $f$ is susceptible  with respect to a specific vulnerability $V$, the framework first activates \texttt{transformer} to transform $f$ into the corresponding trace $T_f$. At the same time, the specification of $V$ is provided as the trace $T_v$. Next the \texttt{matcher} component is called to check whether the trace equation $T_f = T_v$ is satisfiable. If it is satisfiable then a list of solutions $\{V_i\}_{i=1}^n$ is provided. Finally, these solutions are checked by the \texttt{verifiable} for satisfiability. If any of these solutions is satisfied then we conclude that $f$ is vulnerable to $T_v$.

\subsection{Transformation of smart contracts into expressions}

The implementation of \texttt{transformer} takes advantages of the deductive system in Fig.~\ref{fig:syn} which takes a statement $c$ and transforms its into a trace $t$, denoted as $c \triangleright_{n} t$. Here $n$ is an additional parameter that restricts the number of times a function is unfolded, effectively making the transformation terminate on every function.

\begin{figure}
$$
\infer[{[\rname{base}]}]{c_p \trans_{n} c_p}{}
~~~~~
\infer[{[\rname{if}]}]{\ifs{b}{c_1}{c_2} \trans_{n} \textbf{assert}(b) \nxt t_1+\textbf{assert}(!b) \nxt t_2}{c_1 \trans_{n} t_1~~~~ c_2 \trans_{n} t_2}
$$
$$
\infer[{[\rname{while}]}]{\whiles{b}{c} \trans_{n} (\textbf{assert}(b) \nxt t)^* \nxt \textbf{assert}(!b)}{c \trans_{n} t}
~~~~~~
\infer[{[\rname{seq}]}]{c_1;c_2 \trans_{n} t_1 \nxt t_2}{c_1 \trans_{n} t_1~~~~ c_2 \trans_{n} t_2}
$$
$$
\infer[{[\rname{func}_1]}]{c_f \trans_{0} c_f}{}
~~~~~~
\infer[{[\rname{func}_2]}]{c_f \trans_{n} c_f[\{\textit{mod}\} \nxt t]}{n > 0~~\textsf{body}(c_f) = (c,\textit{mod})~~~c \trans_{n-1} t}
$$
\caption{Trace synthesis for the \texttt{transformer} component.}\label{fig:syn}
\end{figure}

