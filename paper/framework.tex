\begin{figure}
	\centering
	\begin{tikzpicture}[% 	
	every text node part/.style={align=center},
	->,
	shorten >=2pt,
	>=stealth,
	node distance=6em,
	com/.style={%
		rectangle,
		rounded corners,
		fill=gray!40,
		minimum width=3em,
		minimum height=1em,
		align=center,
		draw
	},
	sub/.style={%
		rectangle,
		rounded corners,
		minimum width=3em,
		minimum height=1em,
		draw
	},
	data/.style={%
		rectangle,
		minimum width=3em,
		minimum height=1em,
	},
	res/.style={%
		rectangle,
		dashed,
		minimum width=3em,
		minimum height=1em,
		draw
	}
	] 

	\node[com,text width=2cm, text height = 0.3cm] (transformer){Transformer}; 
	\node[com,text width=2cm, text height = 0.3cm] (matcher)[right=of transformer]{Matcher}; 
	\node[com,text width=2cm, text height = 0.3cm] (verifier)[right=of matcher, xshift=5em]{Verifier};   
	\node[data] (input) [above=of transformer,yshift=-2em] {smart contract A};
	\node[data] (trace)[above=of matcher,yshift=-2em]{vunerable trace T}; 
	\node[res] (result)[above=of verifier,yshift=-2em]{report bug};      
	\draw  (input) -- (transformer);    
	\draw  (trace) -- (matcher);   
	\draw (transformer) --node[data,above] {expression E} (matcher);  
	  \path[every node/.style={}]
	  %(verifier) edge[bend left=30] node[below]  {V is infeasible} (matcher)
    %(matcher) edge[bend left=30] node[above]  {matched with new \\ attack vector %V} (verifier)
    (matcher) edge[above] node {attack vectors $\{V_i\}_{i=1}^n$} (verifier)
    (verifier) edge[right] node {some $V_i$ \\ is feasible} (result);
	\end{tikzpicture}
	\caption{Our framework for checking vulnerabilities in Ethereum smart contracts.}\label{fig:framework}	
\end{figure}

\begin{enumerate}
	\item $\textbf{Transformer}$: transform the smart contract A into an expression E (look like regular expression but we can have guard conditions inside).
	\item $\textbf{Matcher}$: check whether $T \in E$, \emph{i.e.} the trace $T$ satisfies the expression $E$. At the same time, generates attack vectors $V$ which contains function/variable assignments.
	\item $\textbf{Verifier}$: run the smart contract with the give input to see whether the vulnerability can be reproduced. We need this step as our semantics over-approximates the Ethereum semantics.
\end{enumerate}

\subsection{Transformation of smart contracts into expressions}

Let $\mathcal{A}$ be the pre-defined abstraction function that maps statements/function's names into abstract symbols. We define the transformation $\trans_n$ that transform a statement into its respective trace form. We use $n$ as the upper limit for the number of times we can unfold nested functions. As a result, our transformation always terminates. Normally, we use $\trans_1$ for all the vulnerability checks, except reentrancy and gasless send in which we use $\trans_2$.
\begin{figure}
$$
\infer[{[\rname{base}]}]{c \trans_{n} c}{
	\begin{array}{c}
	\textsf{primitive}(c)
	\end{array}}
~~~~~
\infer[{[\rname{if}]}]{\ifs{b}{c_1}{c_2} \trans_{n} \textbf{assert}(b);t_1+\textbf{assert}(!b);t_2}{c_1 \trans_{n} t_1~~~~ c_2 \trans_{n} t_2}
$$
$$
\infer[{[\rname{while}]}]{\whiles{b}{c} \trans_{n} (\textbf{assert}(b);t)^*;\textbf{assert}(!b)}{c \trans_{n} t}
~~~~~~
\infer[{[\rname{seq}]}]{c_1;c_2 \trans_{n} t_1 \nxt t_2}{c_1 \trans_{n} t_1~~~~ c_2 \trans_{n} t_2}
$$
$$
\infer[{[\rname{func}_1]}]{c_f \trans_{0} \epsilon}{}
~~~~~~
\infer[{[\rname{func}_2]}]{c_f \trans_{n} c_f[\{I_f\}t]}{n > 0~~\textsf{lookup}(c_f,\chain) = (c,I_f)~~~c \trans_{n-1} t}
$$
\caption{Trace synthesis}\label{fig:syn}
\end{figure}



Matcher: need to propose variables 1st, then check the syntax first, incrementally. From outer to inner. Then semantics.
