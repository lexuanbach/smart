\subsection{A light-weight language for Solidity}
\begin{figure}
$
\begin{array}{lcl}
\textit{msg}&::=~&\textbf{msg}.\textbf{sender}~|~\textbf{msg.value}~|~\textbf{block}.\textbf{number}~|~\textbf{now}~|~\textbf{tx.origin}\\
e&::=~&\textit{msg}~|~\mathbb{Z}~|~\textit{var}~|~\textbf{this}~|~\textit{addr}.\textbf{balance}~|~a[e]~|~a.\textbf{length}~|~e + e~|~ \ldots\\
b&::=~&\textbf{true}~|~\textbf{false}~|~ e = e~|~ e < e ~|~ !b~|~ b \&\& b  ~|~ b || b
\\
c_f&::=~&f(\bar{e})~|~\textit{addr}.f(\bar{e})~|~\textit{addr}.\textbf{call}(f,\bar{e}).\textbf{value}(e)~|~
\\
&&\textit{addr}.\textbf{delegatecall}(f,\bar{e})~|~\textit{addr}.\textbf{send}(e)~|~\textit{addr}.\textbf{transfer}(e)
\\
\textit{exp}&::=~& e~|~b~|~c_f
\\
c_p&::=~&\textbf{skip}~|~x:=\textit{exp}~|~\textbf{continue}~|~\textbf{break}~|~\textbf{return}~e~|~
\textbf{assert}(\textit{exp})~|~\textbf{throw}
\\
c&::=~&c_p~|~c_f~|~c;c~|~\ifs{b}{c}{c}~|~\whiles{b}{c}
\end{array}
$
\caption{A light-weight language for Solidity}\label{fig:lang}
\end{figure}


We adopt the light-weight language in Fig.~\ref{fig:lang} for writing Solidity contracts. In detail, we have the constructor $\textit{msg}$ for global variables such as $\textbf{msg.sender}$---the address of the sender, $\textbf{msg.value}$---the value of the transaction, $\textbf{block.number}$---the current block number, $\textbf{now}$---the current timestamp, and $\textbf{tx.origin}$---the address of the original sender. We use $e$ to write normal expressions, including $\textit{msg}$, integers $\mathbb{Z}$, variables $\textit{var}$,  \textbf{this}---the address of the current contract, \textit{addr}.\textbf{balance}--- the balance of address \textit{addr}, function call $c_f$, array expressions $a[e]$ for index  and $a.\textbf{length}$ for length, together with standard arithmetic operations. On the other hand, we have $b$ for writing standard  Boolean expressions. For simplicity, we assume that our variables only have the following types: address, integer, Boolean and their array counterparts.


The statement constructor $c$ consists of $c_p$ for primitive statements, $c_f$ for function calls, $c;c$ for sequential statements, \textbf{if}-\textbf{else} for conditional statement, and \textbf{while} for loop.  Here primitive statements $c_p$ can be \textbf{skip} (doing nothing), $:=$ for assignment, $\{\textbf{continue},\textbf{break},\textbf{return}\}$ for control flow, $\textbf{assert}$ for checking assertions, and $\textbf{throw}$ for raising exceptions. Meanwhile, functional call $c_f$ contains the normal internal call $f(\bar{e})$ and external call $\textit{addr}.f(\bar{e})$ where $\textit{addr}$ is the callee address, $f$ is the function id, and $\bar{e}$ is the argument list. In addition, $c_f$ also supports other Solidity external calls such as $\textit{addr}.\textbf{call}(f,\bar{e}).\textbf{value}(e)$ that can send $e$ Ethers while invoking the function $f(\bar{e})$ at $\textit{addr}$---the callee address,  $\textit{addr}.\textbf{delegatecall}(f,\bar{e})$ for library calls, or $\textit{addr}.\textbf{transfer}(e)$ and $\textit{addr}.\textbf{send}(e)$ for sending $e$ Ethers to address $\textit{addr}$. The semantics of our language will be elaborated in CITE.
\hide{
. For external calls, the fallback function is executed by default if no function can match the signature $(f,\bar{e})$. Here fallback is a special function in Solidity contracts that has no name and receives no argument. Another external calls is $\textit{addr}.\textbf{call}(f,\bar{e}).\textbf{gas}(e_g).\textbf{value}(e_v)$ which is low-level. By design, low-level calls in Solidity do not propagate exceptions or return values to their callers, instead a Boolean value is returned in which $\textbf{false}$ is equivalent to the occurrence of an exception. When the above $\textbf{call}$ is executed, the current contract sends an amount of $e_v$ Ether to address $\textit{addr}$. At the same time, it also provides a gas stipend of $e_g$ Ether to execute the function $f$ at $\textit{addr}$ with arguments $\bar{e}$. On the other hand, the low-level external call $\textit{addr}.\textbf{delegatecall}(f,\bar{e}).\textbf{gas}(e_g)$ possesses similar behaviors as $\textbf{call}$ except there is no context switching between the caller and the callee. This means the function $f$ is executed in the current contract's context instead of the $\textit{addr}$'s context, thus the field $\textbf{value}$ is redundant and omitted.  The two statements $\textit{addr}.\textbf{send}(e)$ and $\textit{addr}.\textbf{transfer}(e)$ behave similarly as both send an amount of $e$ Ether to address $\textit{addr}$, together with a gas stipend of $2300$ wei to execute the fallback function at address  $\textit{addr}$. However, $\textbf{send}$ is low-level and thus it cannot propagate exceptions like $\textbf{transfer}$.  Finally, we have constructor $c$ for general statements which consists of basic statements $c_b$, function calls $c_f$, composition $c;c$, conditional statement $\ifs{b}{c}{c}$, and while loop $\whiles{b}{c}$.

Note that we focus on the high-level semantics of Solidity and thus our light-weight language do not support the full spectrum of data types and structures in Solidity such as bytes, struct, object, String. For simplicity, we also exclude the statement $\textbf{for}(c_1;b;c_2)\{c\}$ from our language since it can be equivalently transformed into the while statement $c_1;\whiles{b}{\{c;c_2\}}$.
}


\subsection{The trace language}

We introduce the trace language in Fig.~\ref{fig:assert} to specifying vulnerabilities. The constructor $\textit{mod}$ specifies different types of function modifiers supported by Solidity. Next we have P for trace predicates, which are Solidity Boolean expressions enriched with modifiers in  $\textit{mod}$ and quantifiers $\{\forall$, $\exists\}$. Given these premises, the constructor $t$ for trace is defined to be empty $\epsilon$, variable $\bar{t_i}$, primitive statement $c_p$, function call $c_f$, trace with pre-condition $\{P\}t$ or with post-condition $t\{P\}$. Also, we have $c[t]$ to express the scope $t$ of statement $c$ where $t$ is the trace of the function call inside $c$. In detail, scoping statements are limited to normal function call $c_f[t]$, assignment of function call $x := c_f[t]$, and assertion of function call $\textbf{assert}(c_f)[t]$. Our trace constructor also supports standard string operators such as $\nxt$ for concatenation (similar to ;), $+$ for non-deterministic choice, and $*$ (Kleene star) for repetition. For convention, guarded conditions $\{P\}$ is not part of the alphabet. The trace semantics is constructed directly from the Solidity semantics and will be elaborated in CITE.

We say a trace $t$ is \emph{concrete} if it does not contain $+$ or $*$ or trace variables. Otherwise it is called \emph{abstract}. A trace $t$ is \emph{complete} if it is concrete and every statement with function call has a scope. A complete trace $t$ is satisfiable if there exist $\cstate,\cstate'$ and $\kappa_\mathbb{F}$ such that $\conf{t,\cstate} \hookrightarrow \conf{\kappa_{\mathbb{F}},\cstate'}$. We interpret abstract traces as sets of concrete traces according to the following semantics:
$$
\eval{\epsilon} = \{\epsilon,\{P\}\}~~~\eval{\bar{t}} = \mathbb{T}~~~\eval{a} = \{a\}~~~
\eval{\{P\}} = \{\{Q\}~|~Q \Rightarrow P\}
$$
$$
\eval{T_1 \cdot T_2} = \{t_1 \cdot t_2~|~t_i \in \eval{T_i}\}~~~~\eval{a[T]} = \{a[t]~|~t \in \eval{T}\}
$$
$$
\eval{T_1 + T_2} = \eval{T_1} \cup \eval{T_2}
~~~~~~
\eval{T^*} = \{t~|~t = t_1 \ldots t_n \textrm{ where } t_k \in \eval{T}\}
$$

A \emph{trace equation} $E$ has the format $T_1 = T_2$ where $T_1$ and $T_2$ are abstract traces with trace variables. We say a complete trace $t$ is a solution of $E$ if $t \in \eval{T_1} \cap \eval{T_2}$ and $t$ is satisfiable.  

To solve the equation $T_1 = T_2$, we first treat it as a string equation and use SMT solver to retrieve the solutions. During this phase, conditions $\{P\}$ are ignored and thus treated as $\epsilon$. Suppose the complete trace $s$ is the solution, we then insert the conditions back to $s$. If there are two conditions $\{P_1\}$ and $\{P_2\}$ at the same position, we replace it with the conjunction $\{P_1 \wedge P_2\}$. It is easy to verify that the new complete trace $s'$ satisfies $s' \in \eval{T_1} \cap \eval{T_2}$. As a result, it remains to check whether $s'$ is satisfiable using symbolic execution.

\begin{figure}[t]
$
\begin{array}{lcl}
\textit{mod}&::=~&\textsf{owned}~|~ \textsf{public}~|~\textsf{private}~|~\textsf{payable}~|~\textsf{pure}~|~\textsf{view}~|~\textsf{external}\\
P&::=~&\textit{mod}~|~ b ~|~ P \&\& P ~|~ P || P ~|~ !P~|~\exists x.~P~|~ \forall x.~P\\
t&::=~&\epsilon~|~\bar{t_i}~|~c_{p}~|~c_f~|~\{P\}~|~c_f[t]~|~x:=c_f[t]~|~\textbf{assert}(c_f)[t]~|~t \nxt t~|~t+t~|~t^*
\end{array}
$
\caption{The trace language}\label{fig:assert}
\end{figure}

\subsection{Vulnerabilities}

Having described the trace syntax in CITE, we now take a close look on how to use this syntax to define vulnerabilities in Solidity contracts.


Note: we don't support cross-transaction traces, \emph{i.e.} traces with functions executed in different transactions. Need an invariant for state that says "all variables's values are the same", can simply over-approximate to variables in the functions.
\begin{enumerate}
	\item Integer overflow/underflow:
	$$
	\mathcal{F}[\overline{t_1} \nxt \{x \geq \textit{min}\}\overline{t_2}\{x < \textit{min}\} \nxt \overline{t_3}]
	~~~~~
	\mathcal{F}[\overline{t_1} \nxt \{\overline{x} \leq \textit{max}\}\overline{t_2}\{\overline{x} > \textit{max}\} \nxt \overline{t_3}]
	$$
	where $\overline{t_2} \in \{e:= e + e, e:= e - e, e:= e \times e, e:= e~\textbf{div}~e\}$
	\item Unchecked send:
	$$
	\mathcal{F}[\overline{t_1} \nxt \overline{\textit{addr}}.f \nxt \overline{t_2}]
	$$
	
	where $f \in \{\textbf{send},\textbf{transfer},\textbf{call}\}$
	\item Reentrancy:
	$$
	\mathcal{F}[\{\textbf{balance} = \overline{a_0}\}\overline{t_1} \nxt \overline{\textit{addr}}.\textbf{call}.\textbf{value}(\overline{v}).()[\mathcal{F}[\overline{t_2}\{\textbf{balance} < \overline{a_0}\}]] \nxt \overline{t_3}]
	$$
	
	\item Suicide contracts:
	$$
	\mathcal{F}[\{!\textsf{owned}\}\overline{t_1} \nxt \textbf{selfdestruct}(\overline{\textit{addr}}) \nxt \overline{t_2}]
	$$
	\item Time-stamp dependency:
	$$
\mathcal{F}[\overline{t_1} \nxt a \nxt \overline{t_2}]
$$

where $a$ is a primitive statement that contains the $\textbf{now}$ keyword

	\item Block dependency:
	$$
	\mathcal{F}[\overline{t_1} \nxt a \nxt \overline{t_2}]
	$$
	
	where $a$ is a primitive statement that contains the $\textbf{block}.\textbf{number}$ keyword
	
	
	\item Transaction ordering dependency:
	$$
	\overline{\textit{addr}}.\textit{atk}(\{x = \overline{x_0}\}\mathcal{F}_1 \nxt \mathcal{F}_2\{x = \overline{x_1}\}) \cap \{x = \overline{x_0}\}\mathcal{F}_2  \nxt \mathcal{F}_1 \{x = \overline{x_2}\}
	$$
	
	where $x_1 \neq x_2$
	\item Unsafe delegatecall:
	$$
	\mathcal{F}[\overline{t_1} \nxt \overline{\textit{addr}}.\textbf{delegatecall}(\textbf{msg.data}) \nxt \overline{t_2}]
	$$
\end{enumerate}

Let $\lstate$ be the local state and $\gstate$ be the global state. A local configuration is the tuple $\conf{c,\lstate,\gstate}$ of statement, local state and global state. A small step is a binary between two configurations: $\conf{c,\lstate,\gstate} \leadsto \conf{c',\lstate',\gstate'}$. We provide the following sequential operational semantics where $\cstate = \conf{\lstate,\gstate,\chain,b}$:

\begin{figure}
$$
\infer[{[\rname{txs}]}]{\conf{m{::}M,\chain} \leadsto \conf{M,\chain'}}
{
	\begin{array}{c}
	m = (\gstate,\textit{addr})~~~\gstate.\textbf{msg} = (s,v,f,\bar{a})~~~ \chain(\textit{addr}) = (\mathcal{C},\lstate)~~~\mathcal{C}(f,\bar{a}) = c\\
	\chain_1 = \chain[s \stackrel{v}{\rightarrow} \textit{addr}]~~~
	\cstate = (\lstate,\gstate,\chain_1,\textbf{false})~~~
	\conf{c,\cstate} \stackrel{*}{\leadsto} \conf{c_\mathbb{F},\cstate'}~~~
	\chain' = \cstate'.\chain
	\end{array}
}
$$

$$
\infer[{[\rname{txf}]}]{\conf{m{::}M,\chain} \leadsto \conf{M,\chain}}
{
	\begin{array}{c}
	m = (\gstate,\textit{addr})~~~\gstate.\textbf{msg} = (s,v,f,\bar{a})~~~ \chain(\textit{addr}) = (\mathcal{C},\lstate)~~~\mathcal{C}(f,\bar{a}) = c\\
	\chain_1 = \chain[s \stackrel{v}{\rightarrow} \textit{addr}]~~~~~
	\cstate = (\lstate,\gstate,\chain_1,\textbf{false})~~~~~
	\conf{c,\cstate} \stackrel{*}{\leadsto} \conf{\textbf{throw},\cstate'}
	\end{array}
}
$$
\caption{Transaction semantics}\label{fig:tx}
\end{figure}

\subsection{Notations for semantics}

We first introduce several notations to help define the semantics.  A \emph{message} $m$ is a pair $(\gstate,\textit{addr})$ of the message state $\gstate$ and the receiver's address $\textit{addr}$. Here the message state $\gstate$ is represented by the quadruple $(s,v,f,\bar{a})$ that consists of the sender's address $s$, the Ether value $v$ of the transaction, the method id $f$ together with inputs $\bar{a}$ to be called at the receiver's address. A ledger $\chain$ is a finite mapping from addresses to pairs of contract code $\mathcal{C}$ and its state $\lstate$. The method's code in contract $\mathcal{C}$ can be accessed by providing the method id and its parameter, \emph{e.g} $\mathcal{C}(f,\bar{a})$. A transaction step $\stackrel{t}{\leadsto}$ relates two transaction configurations (t-config), \emph{e.g.} $\conf{M,\chain} \stackrel{t}{\leadsto} \conf{M',\chain'}$ where each t-config is a pair of message list and ledger. 

At the contract level we have contract step $\leadsto$ that relates two contract configurations (c-config), \emph{e.g.} $\conf{\kappa,\cstate} \leadsto \conf{\kappa',\cstate'}$ where each c-config is a pair of \emph{activation record} $\kappa$ and \emph{transaction state} $\cstate$. The activation record $\kappa$ can be either a concrete value $v$, a statement $c$, a \emph{scoped statement} $\conf{c}$ (for bookkeeping the scopes of $\textbf{continue}$ and $\textbf{break}$), a pair of activation record and transaction state $\conf{\kappa',\gstate'}$ (for bookkeeping the scopes of function calls), an evaluating assignment $x:= \conf{\kappa',\gstate'}$, or a composition of two activation records $\kappa_1;\kappa_2$. An activation record $\kappa$ is \emph{final} if it is $\textbf{skip}$ or some concrete value $v$. For convenience, we use $\kappa_{\mathbb{F}}$ to represent final activation records. The transaction state $\cstate$ is expressed by the quadruple $(\lstate,\gstate,\chain,b)$ of contract state $\lstate$, message state $\gstate$, ledger $\chain$, and Boolean flag $b$ (for bookkeeping exception propagation).


We also introduce several notations for updating the states. The action $\beta$ for a state $s$ (\emph{i.e.} ledger $\chain$, message $\gstate$, contract $\lstate$) is written as $s[\beta]$. Here action can be updating a variable, \emph{i.e.} $s[x \leftarrow v]$ means the variable $x$ in $s$ is updated with new value $v$. Or the action can be Ether transfer between two addresses, \emph{i.e.} $\chain[\textit{addr}_1 \stackrel{v}{\rightarrow} \textit{addr}_2]$ indicates the transfer of $v$ Ethers from $\textit{addr}_1$ to $\textit{addr}_2$. Also, we use $s.x$ to access the variable $x$ in state $s$, and use $s(x)$ to access its value.

\begin{figure}[t]
\centering
$$
\infer[{[\rname{sk}]}]{\conf{\kappa;\kappa',\cstate} \leadsto \conf{\kappa',\cstate}}{\kappa \in \mathbb{V} \cup \{\textbf{skip},\textbf{continue},\textbf{break}\}}
~~~~~~
\infer[{[\rname{er}]}]{\conf{\kappa,\cstate} \leadsto \conf{\textbf{throw},\cstate}}{\textit{error}}
~~~~~~
\infer[{[\rname{th}]}]{
	\begin{array}{c}
	\conf{\textbf{throw};\kappa,\cstate}
	\leadsto \conf{\textbf{throw},\cstate}
	\end{array}
	}{}
$$

$$
\infer[{[\rname{as}]}]
{
	\begin{array}{l}
	\conf{x:=e,\cstate} 
	\leadsto \conf{\textbf{skip},\cstate'}
	\end{array}
}{
	\begin{array}{c}
	\eval{e}_{\cstate} = v
	~~~
	\cstate' = \cstate[\lstate.x \leftarrow v]
	\end{array}}
~~~
\infer[{[\rname{re}]}]{
	\begin{array}{c}
		\conf{\textbf{return}~e;\kappa, \cstate}
		\leadsto \conf{v,\cstate'}
	\end{array}
}{\eval{e}_\cstate  = v}
~~~
\infer[{[\rname{seq}]}]{
	\begin{array}{c}
		\conf{\kappa_1;\kappa_2,\cstate}
		\leadsto \conf{\kappa'_1;\kappa_2,\cstate'}
	\end{array}
}{\conf{\kappa_1,\cstate} \leadsto \conf{\kappa'_1,\cstate'}}	
$$

$$
\infer[{[\rname{at}]}]{\conf{\textbf{assert}(b),\cstate} \leadsto \conf{\textbf{skip},\cstate}}{
	\begin{array}{c}
	\eval{b}_{\cstate} = \textbf{true}
	\end{array}
}
~~~~~~
\infer[{[\rname{af}]}]{\conf{\textbf{assert}(b),\cstate} \leadsto \conf{\textbf{throw},\cstate}}{
	\begin{array}{c}
	\eval{b}_{\cstate} = \textbf{false}
	\end{array}}
$$

$$
\infer[{[\rname{it}]}]{
	\begin{array}{c}
	\conf{\ifs{b}{c_1}{c_2},\cstate}
	\leadsto \conf{c_1,\cstate}
	\end{array}
}
{
	\begin{array}{c}
	\eval{b}_{\cstate} = \textbf{true}
	\end{array}
}	
~~~~~~
\infer[{[\rname{if}]}]{
	\begin{array}{c}
	\conf{\ifs{b}{c_1}{c_2},\cstate}
	\leadsto \conf{c_2,\cstate}
	\end{array}
}
{
	\begin{array}{c}
	\eval{b}_{\cstate} = \textbf{false}\\
	\end{array}
}	
$$

$$
\infer[{[\rname{wt}]}]
{
	\begin{array}{c}
	\conf{\whiles{b}{c},\cstate}
	\\
	 \leadsto \conf{\conf{c};\whiles{b}{c},\cstate}
	\end{array}
}
{
	\begin{array}{c}
	\eval{b}_{\cstate} = \textbf{true}
	\end{array}
}
~~~~~
\infer[{[\rname{wf}]}]
{
	\begin{array}{c}
	\conf{\whiles{b}{c},\cstate} \\
	\leadsto
	\conf{\textbf{skip},\cstate'}
	\end{array}
}
{
	\begin{array}{c}
	\eval{b}_{\cstate} = \textbf{false}
	\end{array}
}
~~~~
\infer[{[\rname{bk}]}]{
	\begin{array}{c}
	\conf{\conf{\textbf{break};\kappa};\whiles{b}{c'},\cstate} 
	\\
	\leadsto \conf{\textbf{skip},\cstate}
	\end{array}
}{
}
$$

$$
\infer[{[\rname{ct}]}]{
	\begin{array}{c}
	\conf{\conf{\textbf{continue};\kappa},\cstate}
	\leadsto \conf{\textbf{skip},\cstate}
	\end{array}
}{}
~~~~
\infer[{[\rname{in}]}]{\conf{\conf{\kappa_1;\kappa_2},\cstate} \leadsto \conf{\conf{\kappa_1';\kappa_2},\cstate'}}
{
	\begin{array}{c}
	\kappa_1 \not \in \{\textbf{continue},\textbf{break}\}
	\\
	\conf{\kappa_1,\cstate} \leadsto \conf{\kappa_1',\cstate'}
	\end{array}
}
~~~~
\infer[{[\rname{out}]}]{\conf{\conf{\kappa},\cstate} \leadsto \conf{\kappa,\cstate}}{ \kappa \in \mathbb{V} \cup \{\textbf{skip},\textbf{throw}\}}
$$
\caption{Operational semantics for basic commands.}\label{fig:opbasic}
\end{figure}

\begin{figure}[t]
\centering
$$
\infer[{[\rname{ifun}]}]{\conf{f(\bar{a}),\cstate} \leadsto \conf{\conf{c,\cstate'},\cstate}}
{
	\begin{array}{c}	
	\cstate.\mathbb{B}(\eval{\textbf{this}}_{\cstate}) = (\mathcal{C},\lstate_f)~~~~
	\mathcal{C}(f) = (c,\bar{v})
	\\
	\lstate' = \lstate_f[\bar{v} \leftarrow \eval{\bar{a}}_\cstate]
	~~~~
	\cstate' = \cstate[\lstate \leftarrow \lstate'][b \leftarrow \textbf{false}]
	\end{array}
}
~~~~~
\infer[{[\rname{fun}]}]{\conf{\textit{addr}.f(\bar{a}),\cstate} \leadsto \conf{\conf{c,\cstate'},\cstate}}
{
	\begin{array}{c}	
	\cstate.\mathbb{B}(\textit{addr}) = (\mathcal{C},\lstate_f)~~~
	\mathcal{C}(f) = (c,\bar{v})~~~
	\lstate' = \lstate_f[\bar{v} \leftarrow \eval{\bar{a}}_{\cstate}]
	\\
	\gstate' = \cstate.\gstate[\textrm{msg.sender} \leftarrow \eval{\textbf{this}}_{\cstate}][\textrm{msg.value} \leftarrow 0]
	\\
	\cstate' = \cstate[\gstate \leftarrow \gstate'][\lstate \leftarrow \lstate'][b \leftarrow \textbf{false}]
	\end{array}
}
$$
\vspace{-0.5em}
$$
\infer[{[\rname{cal}]}]{\conf{\textit{addr}.\textbf{call}(f,\bar{a}).\textbf{value}(e),\cstate} \leadsto \conf{\conf{c,\cstate'},\cstate}}
{
	\begin{array}{c}	
	\cstate.\mathbb{B}(\textit{addr}) = (\mathcal{C},\lstate_f)
	~~~
	\mathcal{C}(f) = (c,\bar{v})~~~
	\lstate' = \lstate_f[\bar{v} \leftarrow \eval{\bar{a}}_{\cstate}]
	\\
	\gstate' = \cstate.\gstate[\textrm{msg.sender} \leftarrow \eval{\textbf{this}}_{\cstate}][\textrm{msg.value} \leftarrow \eval{e}_{\cstate}]
	\\
	\chain' = \cstate.\chain[\eval{\textbf{this}}_{\cstate} \stackrel{\eval{e}_{\cstate}}{\rightarrow} \textit{addr}]
	\\
	\cstate' = \cstate[\chain \leftarrow \chain'][\gstate \leftarrow \gstate'][\lstate \leftarrow \lstate'][b \leftarrow \textbf{true}]
	\end{array}
}
~~~~
\infer[{[\rname{tf}]}]{\conf{\textit{addr}.\textbf{transfer}(e),\cstate} \leadsto \conf{\conf{c,\cstate'},\cstate}}
{
	\begin{array}{c}	
	\cstate.\mathbb{B}(\textit{addr}) = (\mathcal{C},\lstate_f)~~~~
	\mathcal{C}(\_) = (c,\_)
	\\
	\gstate' = \cstate.\gstate[\textrm{msg.sender} \leftarrow \eval{\textbf{this}}_{\cstate}][\textrm{msg.value} \leftarrow \eval{e}_{\cstate}]
		\\
	\chain' = \cstate.\chain[\eval{\textbf{this}}_{\cstate} \stackrel{\eval{e}_{\cstate}}{\rightarrow} \textit{addr}]
	\\
	\cstate' = \cstate[\chain \leftarrow \chain'][\gstate \leftarrow \gstate'][\lstate \leftarrow \lstate_f][b \leftarrow \textbf{false}]
	\end{array}
}
$$
$$
\infer[{[\rname{send}]}]{
	\begin{array}{c}
	\conf{\textit{addr}.\textbf{send}(e),\cstate} \leadsto
	\\
	\conf{\textit{addr}.\textbf{call}.\textbf{value}(e)(,),\cstate}
	\end{array}
}
~~~~~~~~
\infer[{[\rname{dcal}]}]{\conf{\textit{addr}.\textbf{delegatecall}(f,\bar{a}),\cstate} \leadsto \conf{\conf{c,\cstate'},\cstate}}
{
	\begin{array}{c}	
	\cstate.\mathbb{B}(\textit{addr}) = (\mathcal{C},\lstate_f)
	~~~\mathcal{C}(f) = (c,\bar{v})
	\\
	\lstate' = \lstate_f[\bar{v} \leftarrow \eval{\bar{a}}_{\cstate}]
	~~~
	\cstate' = \cstate[\lstate \leftarrow \lstate'][b \leftarrow \textbf{true}]
	\end{array}
}
$$
$$
\infer[{[\rname{kill}]}]{
	\conf{\textbf{selfdestruct}(\textit{addr}),\cstate} \leadsto \conf{\textbf{skip},\cstate'}
}
{
	\begin{array}{ll}
	\chain' = \cstate.\chain[\eval{\textbf{this}}_{\cstate} \stackrel{e}{\rightarrow} \textit{addr}][\eval{\textbf{this}}_\cstate \leftarrow \textbf{None}]
	\\
	\eval{\textbf{this}.\textrm{balance}}_{\cstate} = e
	~~~~
	\cstate' = \cstate[\chain \leftarrow \chain']
	\end{array}
}
~~~~~
\infer[{[\rname{run}]}]{\conf{\conf{\kappa,\cstate_f},\cstate} \leadsto \conf{\conf{\kappa',\cstate_f'},\cstate}}
{
	\begin{array}{c}	
	\conf{\kappa,\cstate_f} \leadsto \conf{\kappa',\cstate_f'}
	\end{array}
}
$$
$$
\infer[{[\rname{cf}]}]{\conf{\conf{\textbf{throw},\cstate_f},\cstate} \leadsto \conf{\textbf{false},\cstate}}
{
	\begin{array}{c}	
	\cstate.b = \textbf{true}
	\end{array}
}
~~~~
\infer[{[\rname{ff}]}]{\conf{\conf{\textbf{throw},\cstate_f},\cstate} \leadsto \conf{\textbf{throw},\cstate}}
{
	\begin{array}{c}	
	\cstate.b = \textbf{false}
	\end{array}
}
$$
\vspace{-0.5em}
$$
\infer[{[\rname{fs}]}]{\conf{\conf{\kappa_\mathbb{F},\cstate_f},\cstate} \leadsto \conf{\kappa_\mathbb{F},\cstate'}}
{
	\begin{array}{c}
	\cstate.b = \textbf{false}~~~~
	\cstate_f.\chain(\eval{\textbf{this}}_{\cstate_f}) = (\mathcal{C},\lstate_f)
	\\
	\chain' = \cstate_f.\chain[\eval{\textbf{this}}_{\cstate_f} \leftarrow (\mathcal{C},\cstate_f.\lstate)]
	~~~~
	\cstate' = \cstate[\chain \leftarrow \chain']
	\end{array}
}
~~~~~~
\infer[{[\rname{cs}]}]{\conf{\conf{\kappa_\mathbb{F},\cstate_f},\cstate} \leadsto \conf{\textbf{true},\cstate'}}
{
	\begin{array}{c}
	\cstate.b = \textbf{true}~~~~
\cstate_f.\chain(\eval{\textbf{this}}_{\cstate_f}) = (\mathcal{C},\lstate_f)
\\
\chain' = \cstate_f.\chain[\eval{\textbf{this}}_{\cstate_f} \leftarrow (\mathcal{C},\cstate_f.\lstate)]
~~~~
\cstate' = \cstate[\chain \leftarrow \chain']
	\end{array}
}
$$
$$
\infer[\rname{eas}_1]{\conf{x:=c_f,\cstate} \leadsto \conf{x:=\conf{c,\cstate'},\cstate''}}
{
\conf{c_f,\cstate} \leadsto \conf{\conf{\kappa,\cstate'},\cstate''}
}
~~~~~
\infer[\rname{eas}_2]{\conf{x:=\conf{v,\cstate},\cstate'} \leadsto \conf{x:= v,\cstate''}}
{
\conf{\conf{v,\cstate},\cstate'} \leadsto \conf{v,\cstate''}
}
~~~~~
\infer[\rname{eas}_3]{\conf{x:=\conf{\textbf{throw},\cstate},\cstate'} \leadsto \conf{\textbf{throw},\cstate''}}
{
	\conf{\conf{\textbf{throw},\cstate},\cstate'} \leadsto \conf{\textbf{throw},\cstate''}
}
$$
\caption{Operational semantics for function calls.}\label{fig:opfun}
\end{figure}

\begin{figure}
$$
\infer[{[\rname{base}]}]{\conf{\cstate,c,\cstate'} \models \mathcal{A}(c)}{\textit{primitive}(c)~~~\conf{c,\cstate} \leadsto \conf{v,\cstate'}}
~~~~~
\infer[{[\rname{em}]}]{\conf{\cstate,c,\cstate'} \models t}{\conf{c,\cstate} \stackrel{\epsilon}{\leadsto} \conf{c_1,\cstate_1}~~~\conf{\cstate_1,c_1,\cstate'} \models t}
$$

$$
\infer[{[\rname{case}_1]}]{\conf{\cstate,c,\cstate'} \models t_1+t_2}{\conf{\cstate,c,\cstate'} \models t_1}
~~~~~
\infer[{[\rname{case}_2]}]{\conf{\cstate,c,\cstate'} \models t_1+t_2}{\conf{\cstate,c,\cstate'} \models t_2}
$$

$$
\infer[{[\rname{cut}]}]{\conf{\cstate,c_1;c_2;\cstate'} \models t_1 \nxt t_2}{\conf{\cstate,c_1,\cstate_1} \models t_1~~~\conf{\cstate_1,c_2,\cstate'} \models t_2}
~~~~~
\infer[{[\rname{star}_1]}]{\conf{\cstate,c,\cstate'} \models t^*}{\conf{\cstate,c,\cstate'} \models \epsilon}
~~~~~
\infer[{[\rname{star}_2]}]{\conf{\cstate,c,\cstate'} \models t^*}{\conf{\cstate,c,\cstate'} \models t \nxt t^*}
$$

$$
\infer[{[\rname{cond}_1]}]{\conf{\cstate,c,\cstate'} \models \{P\}t}{\conf{\cstate,c,\cstate'} \models t~~~~~\eval{P}_{\cstate} = \textbf{true}}
~~~~~
\infer[{[\rname{cond}_2]}]{\conf{\cstate,c,\cstate'} \models t\{P\}}{\conf{\cstate,c,\cstate'} \models t~~~~~\eval{P}_{\cstate'} = \textbf{true}}
$$



$$
\infer[{[\rname{fun}_1]}]{\conf{\cstate,f,\cstate'} \models \mathcal{F}(f)[t]}
{
	\begin{array}{ll}
	\conf{f,\cstate} \leadsto \conf{\conf{c_f,\cstate_f},\cstate'}
	&~~~\conf{c_f,\cstate_f} \stackrel{*}{\leadsto} \conf{v,\cstate_f'}
	\\
	\conf{\conf{v,\cstate_f'},\cstate} \leadsto \conf{v,\cstate'}
	&~~~\conf{\cstate_f,c_f,\cstate_f'} \models t
	\end{array}
}
~~~~~
\infer[{[\rname{fun}_2]}]{\conf{\cstate,f,\cstate'} \models \mathcal{F}(f)}{\conf{f,\cstate} \stackrel{*}{\leadsto} \conf{v,\cstate'}}
$$
\caption{Trace semantics}\label{fig:trace}
\end{figure}


A trace $t$ is belong to a program $c$, denoted as $t \in \mathbb{T}(c)$, iff for all $\cstate$, $\cstate'$ we have $\conf{t,\cstate} \stackrel{*}{\leadsto} \conf{f,\cstate'}$ implies $\conf{c,\cstate} \stackrel{*}{\leadsto} \conf{f,\cstate'}$ where $\conf{f,\cstate'}$ is final, \emph{i.e.}:
$$
t \in \mathbb{T}(c) ~\stackrel{\triangle}{=}~ \forall \cstate,\cstate',t_{\mathbb{F}}, f.\conf{t,\cstate} \hookrightarrow \conf{t_{\mathbb{F}},\cstate'} \Rightarrow \conf{c,\cstate} \stackrel{*}{\leadsto} \conf{t_{\mathbb{F}},\cstate'}
$$


Evaluation:
\begin{figure}
$$
\infer[\rname{base}]{\conf{t,\cstate} \hookrightarrow \conf{t_{\mathbb{F}},\cstate'}}{\textsf{command}(t)~~~\conf{t,\cstate} \stackrel{*}{\leadsto} \conf{t_{\mathbb{F}},\cstate'}}
~~~
\infer[\rname{left}]{\conf{t_1+t_2,\cstate} \hookrightarrow \conf{t_{\mathbb{F}},\cstate'}}{\conf{t_1,\cstate} \hookrightarrow \conf{t_{\mathbb{F}},\cstate'}}
~~~
\infer[\rname{right}]{\conf{t_1+t_2,\cstate} \hookrightarrow \conf{t_{\mathbb{F}},\cstate'}}{\conf{t_2,\cstate} \hookrightarrow \conf{t_{\mathbb{F}},\cstate'}}
$$

$$
\infer[\rname{pre}]{\conf{\{P\}t,\cstate} \hookrightarrow \conf{t_{\mathbb{F}},\cstate'}}{\eval{P}_\cstate = \textbf{true}~~~\conf{t,\cstate} \hookrightarrow \conf{t_{\mathbb{F}},\cstate'}}
~~~~~
\infer[\rname{post}]{\conf{t\{P\},\cstate} \hookrightarrow \conf{t_{\mathbb{F}},\cstate'}}{\conf{t,\cstate} \hookrightarrow \conf{t_{\mathbb{F}},\cstate'}~~~\eval{P}_{\cstate'} = \textbf{true}}
$$

$$
\infer[\rname{star}_1]{\conf{t^*,\cstate} \hookrightarrow \conf{t_{\mathbb{F}},\cstate'}}{\conf{t \nxt t^*,\cstate} \hookrightarrow \conf{t_{\mathbb{F}},\cstate'}}
~~~~~
\infer[\rname{star}_2]{\conf{t^*,\cstate} \hookrightarrow \conf{\textbf{skip},\cstate}}{}
$$

$$
\infer[\rname{seq}_1]{\conf{t_1 \nxt t_2,\cstate} \hookrightarrow \conf{t_{\mathbb{F}}' ,\cstate'}}{\conf{t_1,\cstate} \hookrightarrow \conf{t_{\mathbb{F}},\cstate_1}~~~\conf{t_2,\cstate_1} \hookrightarrow \conf{t_{\mathbb{F}}',\cstate'}~~~t_\mathbb{F} \neq \textbf{throw}}
~~~~~
\infer[\rname{seq}_2]{\conf{t_1 \nxt t_2,\cstate} \hookrightarrow \conf{\textbf{throw} ,\cstate'}}{\conf{t_1,\cstate} \hookrightarrow \conf{\textbf{throw},\cstate'}}
$$

$$
\infer[\rname{scope}]{\conf{t_f[t],\cstate} \hookrightarrow \conf{t'_{\mathbb{F}},\cstate'}}
{
	\begin{array}{c}
	\conf{t_f,\cstate} \leadsto \conf{\conf{c_f,\cstate_f},\cstate_1}~~~\conf{t,\cstate_f} \hookrightarrow \conf{t_{\mathbb{F}},\cstate_f'}\\
	\conf{\conf{t_{\mathbb{F}},\cstate_f'},\cstate_1} \leadsto \conf{t'_{\mathbb{F}},\cstate'}~~~~~~t_f \in \{f(\bar{a}),\textit{addr}.f(\bar{a})\}
	\end{array}
}
$$
\caption{A trace semantics}\label{fig:tracesem}
\end{figure}



Note that $\gas(c,\lstate,\gstate)$ is a partial function and only works if $\gstate.\textbf{gas} \geq 0$. We don't model msg.sig and msg.data as they are measured in bytes which is not meaningful for high-level reasoning. We also assume that delegatecall will either return true or false which is similar to normal call.
\subsection{Semantics}

Let 

\subsection{Examples}